
\documentclass[a4paper,12pt]{article}
\usepackage{graphicx}
\usepackage[table,xcdraw]{xcolor}
\usepackage{geometry}
\usepackage{float}
\usepackage[colorlinks=false, hidelinks]{hyperref}
\usepackage{fancyhdr}% For page numbering
\geometry{top=1in,bottom=1in,left=1in,right=1in}
\usepackage{listings}
\usepackage{xcolor}
\usepackage{hyperref}
\pagestyle{empty}


\begin{document}

\begin{center}
    \vspace{0.2cm}
    \textbf{\large{Course Title: Software Engineering \& ISD Lab}}\\
    \vspace{0.2cm}
    \textbf{Course Code: CSE-404}\\
    \vspace{0.2cm}
    \textbf{4\textsuperscript{th}Year 1\textsuperscript{st}Semester Examination 2023}\\
    \vspace{0.5cm}
    \textbf{Date of Submission: \today}\\

    \vspace{1.5cm}
    \includegraphics[width=0.35\textwidth]{images/logo.png}\\ % Replace 'logo.png' with the correct path if you have the university logo image
    \vspace{1cm}

    \textbf{Submitted to}\\
    \vspace{0.2cm}
    \textbf{\href{https://juniv.edu/teachers/musfique.anwar}{Dr. Md Musfique Anwar}}\\
    {Professor}\\
    \vspace{0.2cm}
    \textbf{\href{https://juniv.edu/teachers/hkabir}{Dr. Md. Humayun Kabir}}\\
    {Professor}\\


    \vspace{1cm}

    \begin{table}[h!]
        \centering
        \arrayrulecolor{black}
        \begin{tabular}{|c|c|c|c|}
            \hline
            \rowcolor[HTML]{2F4F4F} % Changed header background color to dark slate gray
            {\color[HTML]{FFFFFF}\textbf{Sl}}& {\color[HTML]{FFFFFF}\textbf{Class Roll}}& {\color[HTML]{FFFFFF}\textbf{Exam Roll}}& {\color[HTML]{FFFFFF}\textbf{Name}}\\ \hline
            \rowcolor[HTML]{B0E0E6}
            \textbf{1}& \textbf{408} & \textbf{202220} & \textbf{Sudipta Singha} \\ \hline
       
        \end{tabular}
    \end{table}

    \vspace{1cm}

    Department of Computer Science and Engineering\\
    Jahangirnagar University\\
    Savar, Dhaka, Bangladesh\\
\end{center}

\newpage

\tableofcontents

\newpage
\pagestyle{fancy}
\fancyhf{}
\fancyfoot[C]{\thepage} % Page number in the center of the footer
\section{Introduction}
The Jahangirnagar University Medical Center Management System project was designed
to streamline and digitalize the management of the university’s medical center. This sys-
tem aimed to provide essential features such as user registration, appointment scheduling,
inventory tracking for the medical store, and the publication of informational vlogs on
seasonal diseases. The project’s primary objective was to enhance efficiency, improve
accessibility for users, and ensure better management of medical resources. During my
involvement, the project achieved several milestones, including a functional user authen-
tication system, a dynamic dashboard for doctors, lab technician, storekeeper users, and
the successful integration of ambulance driver details for emergency services. The project
utilized Django as the backend framework, ensuring a robust and scalable system. .
Github link for the project is provided here
\begin{figure}[H]
    \centering
    \includegraphics[width=0.5\textwidth]{images/simple_prcode.png}
    \caption{QR code for the project link}
    \label{fig:qrcode}
\end{figure}
\textbf{Features}
\begin{itemize}
    \item Sign Up
    \item Log In
    \item See the information of waiting patients
    \item Prescribe medicines and test for a patient
    \item Dispense medicines to patients
    \item Update stock information
    \item Certify patients for fund collection
    \item Publish information and preventative measures for seasonal diseases
    \item Schedule appointment for doctors and lab tests
    \item Collect test reports and Prescriptions
    \item Reschedule test appointments
    \item Submit test reports
    \item Visit the seasonal diseases portal
\end{itemize}
\section{Sprint Overview}
In Sprint 2, we aimed to utilize the feedback and lessons learned from Sprint 1 to improve both the product
and our agile scrum processes. This included updating our product backlog, planning a new sprint backlog, and
incorporating Test-Driven Development (TDD) and continuous integration testing into our workflow. These
improvements were intended to enhance the quality of our deliverables and streamline our development
practices.
\section{Practice Improvement}
\end{document}
